
This appendix specifies precisely what programmer visible state GLUT
maintains.  There are three categories of programmer visible state that
GLUT maintains:  global, window, and menu.  The window and menu state
categories are maintained for each created window or menu.  
Additional overlay-related window state is maintained when an overlay
is established for a window for the lifetime of the overlay.

The tables below name each element of state, define its type, specify
what GLUT API entry points set or change the state (if possible),
specify what GLUT API entry point or \rndx{glutGet},
\rndx{glutDeviceGet}, or \rndx{glutLayerGet} state constant is used to
get the state (if possible), and how the state is initially set.  For
details of how any API entry point operates on the specified state, see
the routine's official description.  Footnotes for each category of
state indicate additional caveats to the element of state.

\subsection{Types of State}

These types are used to specify GLUT's programmer visible state:
\begin{description}

\item[Bitmask]
A group of boolean bits.

\item[Boolean]
True or false.

\item[Callback]
A handle to a user-supplied routine invoked when the given callback is triggered
(or {\tt NULL} which is the default callback).

\item[ColorCell]
Red, green, and blue color component triple, an array of which makes
a colormap.

\item[Cursor]
A GLUT cursor name.

\item[Integer]
An integer value.

\item[Layer]
Either normal plane or overlay.

\item[MenuItem]
Either a menu entry or a submenu trigger.  Both subtypes contain
of a {\em String} name.  A menu entry has an {\em Integer} value.  A
submenu cascade has an {\em Integer} menu name naming its associated submenu.

\item[MenuState]
Either in use or not in use.

\item[Stacking]
An ordering for top-level windows and sub-windows having the same
parent.  Higher windows obscure lower windows.

\item[State]
One of shown, hidden, or iconified.

\item[String]
A string of ASCII characters.

\item[Timer]
A triple of a timer {\em Callback}, an {\em Integer} callback parameter, and a time
in milliseconds (that expires in real time).

\end{description}

\subsection{Global State}

There are two types of global state:  program controlled state which can be
modified directly or indirectly by the program, and fixed system dependent state.

\subsubsection{Program Controlled State}

{\footnotesize
\begin{tabular}{|l|l|l|l|l|} \hline
Name & Type & Set/Change & Get & Initial \\ \hline
currentWindow & Integer & glutSetWindow \takeNote & glutGetWindow & 0 \\
currentMenu & Integer & glutSetMenu \takeNote & glutGetMenu & 0 \\
initWindowX & Integer & glutInitWindowPosition & GLUT\_INIT\_WINDOW\_X & -1 \\
initWindowY & Integer & glutInitWindowPosition & GLUT\_INIT\_WINDOW\_Y & -1 \\
initWindowWidth & Integer & glutInitWindowSize & GLUT\_INIT\_WINDOW\_WIDTH & 300 \\
initWindowHeight & Integer & glutInitWindowSize & GLUT\_INIT\_WINDOW\_HEIGHT & 300 \\
initDisplayMode & Bitmask & glutInitDisplayMode & GLUT\_INIT\_DISPLAY\_MODE & GLUT\_RGB, \\
& & & & GLUT\_SINGLE, \\
& & & & GLUT\_DEPTH \\
idleCallback & Callback & glutIdleFunc & - & NULL \\
menuState & MenuState & - & \takeNote & NotInUse \\
menuStateCallback & Callback & glutMenuEntryFunc & - & NULL \\
timerList & list of Timer & glutTimerFunc & - & none \\
\hline
\end{tabular}
}

\resetNote

{\footnotesize
\begin{description}
\itemsep 0in
\item[\takeNote] The {\em currentWindow} is also changed implicitly by every window
or menu
callback (to the window triggering the callback) and the creation of a window
(to the window being created).
\item[\takeNote] The {\em currentMenu} is also changed implicitly by every menu
callback (to the menu triggering the callback) and the creation of a menu
(to the menu being created).
\item[\takeNote] The menu state callback is triggered when the menuState
changes.
\end{description}
}

\subsubsection{Fixed System Dependent State}

{\footnotesize
\begin{tabular}{|l|l|l|l|l|} \hline
Name & Type &  Get \\ \hline
screenWidth & Integer & GLUT\_SCREEN\_WIDTH \\
screenHeight & Integer & GLUT\_SCREEN\_HEIGHT \\
screenWidthMM & Integer & GLUT\_SCREEN\_WIDTH\_MM \\
screenHeightMM & Integer & GLUT\_SCREEN\_HEIGHT\_MM \\
hasKeyboard & Boolean & GLUT\_HAS\_KEYBOARD \\
hasMouse & Boolean & GLUT\_HAS\_MOUSE \\
hasSpaceball & Boolean & GLUT\_HAS\_SPACEBALL \\
hasDialAndButtonBox & Boolean & GLUT\_HAS\_DIAL\_AND\_BUTTON\_BOX \\
hasTablet & Boolean & GLUT\_HAS\_TABLET \\
numMouseButtons & Integer & GLUT\_NUM\_MOUSE\_BUTTONS \\
numSpaceballButtons & Integer & GLUT\_NUM\_SPACEBALL\_BUTTONS \\
numButtonBoxButtons & Integer & GLUT\_NUM\_BUTTON\_BOX\_BUTTONS \\
numDials & Integer & GLUT\_NUM\_DIALS \\
numTabletButtons & Integer & GLUT\_NUM\_TABLET\_BUTTONS \\
\hline
\end{tabular}
}

\subsection{Window State}

For the purposes of listing the window state elements, window state is
classified into three types:  base state, frame buffer capability state,
and layer state.
The tags {\em top-level}, {\em sub-win}, and {\em cindex} indicate the
table entry applies only to top-level windows, subwindows, or color index
windows respectively.

\subsubsection{Basic State}

\resetNote
{\footnotesize
\begin{tabular}{|l|l|l|l|l|} \hline
Name & Type & Set/Change & Get & Initial \\ \hline
number & Integer & - & glutGetWindow & {\em top-level:} glutCreateWindow \takeNote \\
& & & & {\em sub-win:} glutCreateSubWindow \sameNote \\
x & Integer & glutPositionWindow & GLUT\_WINDOW\_X & {\em top-level:} initWindowX \takeNote \\
& & & & {\em sub-win:} glutCreateSubWindow \\
y & Integer & glutPositionWindow & GLUT\_WINDOW\_Y & {\em top-level:} initWindowY \takeNote \\
& & & & {\em sub-win:} glutCreateSubWindow \\
width & Integer & glutReshapeWindow & GLUT\_WINDOW\_WIDTH & {\em top-level:} initWindowWidth \takeNote \\
& & & & {\em sub-win:} glutCreateSubWindow \\
height & Integer & glutReshapeWindow & GLUT\_WINDOW\_HEIGHT & {\em top-level:} initWindowHeight \takeNote \\
& & & & {\em sub-win:} glutCreateSubWindow \\
top-level: fullScreen & Boolean & glutFullScreen & & False \\
& & glutPositionWindow & & \\
& & glutReshapeWindow \takeNote & & \\
cursor & Cursor & glutSetCursor & GLUT\_WINDOW\_CURSOR & GLUT\_CURSOR\_INHERIT \\
stacking & Stacking & glutPopWindow & - & top \\
& & glutPushWindow & & \\
displayState & State \takeNote & glutShowWindow \takeNote & - & shown \\
& & glutHideWindow & & \\
& & glutIconifyWindow & & \\
visibility & Visibility & \takeNote & \takeNote & undefined \\
redisplay & Boolean & glutPostRedisplay \takeNote & - & False \\
top-level: windowTitle & String & glutWindowTitle & - & glutCreateWindow \\
top-level: iconTitle & String & glutIconTitle & - & glutCreateWindow \\
displayCallback & Callback & glutDisplayFunc & - & NULL \takeNote \\
reshapeCallback & Callback & glutReshapeFunc & - & NULL \takeNote \\
keyboardCallback & Callback & glutKeyboardFunc & - & NULL \\
mouseCallback & Callback & glutMouseFunc & - & NULL \\
motionCallback & Callback & glutMotionFunc & - & NULL \\
passiveMotionCallback & Callback & glutPassiveMotionFunc & - & NULL \\
specialCallback & Callback & glutSpecialFunc & - & NULL \\
spaceballMotionCallback & Callback & glutSpaceballMotionFunc & - & NULL \\
spaceballRotateCallback & Callback & glutSpaceballRotateFunc & - & NULL \\
spaceballButtonCallback & Callback & glutSpaceballButtonFunc & - & NULL \\
buttonBoxCallback & Callback & glutButtonBoxFunc & - & NULL \\
dialsCallback & Callback & glutDialsFunc & - & NULL \\
tabletMotionCallback & Callback & glutTabletMotionFunc & - & NULL \\
tabletButtonCallback & Callback & glutTabletButtonFunc & - & NULL \\
visibilityCallback & Callback & glutVisibilityFunc & - & NULL \\
entryCallback & Callback & glutEntryFunc & - & NULL \\
{\em cindex:} colormap & array of & glutSetColor & glutGetColor & undefined \\
& ColorCell & glutCopyColormap & & \\
windowParent & Integer & - & GLUT\_WINDOW\_PARENT & {\em top-level:} 0 \\
& & & & {\em sub-win:} \takeNote \\
numChildren & Integer & glutCreateSubWindow & GLUT\_NUM\_CHILDREN & 0 \\
& & glutDestroyWindow & & \\
leftMenu & Integer & glutAttachMenu & - & 0 \\
& & glutDetachMenu & & \\
middleMenu & Integer & glutAttachMenu & - & 0 \\
& & glutDetachMenu & & \\
rightMenu & Integer & glutAttachMenu & - & 0 \\
& & glutDetachMenu & & \\
\hline
\end{tabular}
}

\resetNote

{\footnotesize
\begin{description}
\itemsep 0in
\item[\takeNote]  Assigned dynamically from unassigned window numbers greater than zero.
\item[\takeNote]  If {\em initWindowX} is greater or equal to zero {\em and}
{\em initWindowY} is greater or equal to zero then {\em initWindowX}, else
window location left to window system to decide.
\item[\takeNote]  If {\em initWindowY} is greater or equal to zero {\em and}
{\em initWindowX} is greater or equal to zero then {\em initWindowY}, else
window location left to window system to decide.
\item[\takeNote]  If {\em initWindowWidth} is greater than zero {\em and} {\em initWindowHeight} is greater than zero
the {\em initWindowWidth}, else
window size left to window system to decide.
\item[\takeNote]  If {\em initWindowHeight} is greater than zero {\em and} {\em initWindowWidth} is greater than zero
then {\em initWindowHeight}, else
window size left to window system to decide.
\item[\takeNote]  {\tt glutFullScreen} sets to true; {\tt glutPositionWindow} and
{\tt glutReshapeWindow} set to false.
\item[\takeNote]  Subwindows can not be iconified.
\item[\takeNote]  Window system events can also change the displayState.
\item[\takeNote]  Visibility of a window can change for window system dependent reason, for
  example, a new window may occlude the window.  {\tt glutPopWindow} and {\tt glutPushWindow} can
  affect window visibility as a side effect.
\item[\takeNote]  The visibility callback set by {\tt glutVisibilityFunc} allows the visibility state to be tracked.
\item[\takeNote]  The redisplay state can be explicitly enabled by {\tt glutRedisplayFunc} or
  implicitly in response to normal plane redisplay events from the window system.
\item[\takeNote]  A window's {\em displayCallback} {\em must} be registered before the first display
  callback would be triggered (or the program is terminated).
\item[\takeNote]  Instead of being a no-op as most {\tt NULL} callbacks are, a {\tt NULL} {\em reshapeCallback}
sets the OpenGL viewport to render into the complete window, i.e., {\tt glViewport(0,0,width, height)}.
\item[\takeNote]  Determined by {\em currentWindow} at {\tt glutCreateSubWindow} time.
\end{description}
}

\subsubsection{Frame Buffer Capability State}
\label{frame-buffer-state}

\resetNote
{\footnotesize
\begin{tabular}{|l|l|l|} \hline
Name & Type & Get \\ \hline
Total number of bits in color buffer & Integer & GLUT\_WINDOW\_BUFFER\_SIZE \\
Number of bits in stencil buffer & Integer & GLUT\_WINDOW\_STENCIL\_SIZE \\
Number of bits in depth buffer & Integer & GLUT\_WINDOW\_DEPTH\_SIZE \\
Number of bits of red stored in color buffer & Integer & GLUT\_WINDOW\_RED\_SIZE \\
Number of bits of green stored in color buffer & Integer & GLUT\_WINDOW\_GREEN\_SIZE \\
Number of bits of blue stored in color buffer & Integer & GLUT\_WINDOW\_BLUE\_SIZE \\
Number of bits of alpha stored in color buffer & Integer & GLUT\_WINDOW\_ALPHA\_SIZE \\
Number of bits of red stored in accumulation buffer & Integer & GLUT\_WINDOW\_ACCUM\_RED\_SIZE \\
Number of bits of green stored in accumulation buffer & Integer & GLUT\_WINDOW\_ACCUM\_GREEN\_SIZE \\
Number of bits of blue stored in accumulation buffer & Integer & GLUT\_WINDOW\_ACCUM\_BLUE\_SIZE \\
Number of bits of alpha stored in accumulation buffer & Integer & GLUT\_WINDOW\_ACCUM\_ALPHA\_SIZE \\
Color index colormap size & Integer & GLUT\_WINDOW\_COLORMAP\_SIZE \\
If double buffered & Boolean & GLUT\_WINDOW\_DOUBLEBUFFER \\
If RGBA color model & Boolean & GLUT\_WINDOW\_RGBA \\
If stereo & Boolean & GLUT\_WINDOW\_STEREO \\
Number of samples for multisampling & Integer & GLUT\_WINDOW\_MULTISAMPLE \\
\hline
\end{tabular}
}

A window's (normal plane) frame buffer capability state is derived from
the global initDisplayMode state at the window's creation.  A window's
frame buffer capabilities can not be changed.

\subsubsection{Layer State}

\resetNote
{\footnotesize
\begin{tabular}{|l|l|l|l|l|} \hline
Name & Type & Set/Change & Get & Initial \\ \hline
\resetNote
hasOverlay & Boolean & glutEstablishOverlay & GLUT\_HAS\_OVERLAY & False \\
& & glutRemoveOverlay & & \\
overlayPossible & Boolean & \takeNote & GLUT\_OVERLAY\_POSSIBLE & False \\
layerInUse & Layer & glutUseLayer \takeNote & GLUT\_LAYER\_IN\_USE & normal plane \\
{\em cindex:} transparentIndex & Integer & - & GLUT\_TRANSPARENT\_INDEX & \takeNote \\
overlayRedisplay & Boolean & glutPostOverlayRedisplay \takeNote & - & False \\
overlayDisplayCallback & Callback & glutOverlayDisplayFunc & - & NULL \\
overlayDisplayState & State & glutShowOverlay & - & shown \\
                    &       & glutHideOverlay &   &       \\
normalDamaged & Boolean & \takeNote & GLUT\_NORMAL\_DAMAGED & False \\
overlayDamaged & Boolean & \takeNote & GLUT\_OVERLAY\_DAMAGED & False \\
\hline
\end{tabular}
}

\resetNote

{\footnotesize
\begin{description}
\itemsep 0in
\item[\takeNote]  Whether an overlay is possible is based on the {\em initDisplayMode} state {\em
   and} the frame buffer capability state of the window.
\item[\takeNote]  The {\em layerInUse} is implicitly set to overlay after {\tt glutEstablishOverlay};
  likewise, {\tt glutRemoveOverlay} resets the state to normal plane.
\item[\takeNote]  The {\em transparentIndex} is set when a color index overlay is established.  It
   cannot be set; it may change if the overlay is re-established.  When no overlay is in use
   or if the overlay is not color index, the {\em transparentIndex} is -1.
\item[\takeNote]  The {\em overlayRedisplay} state can be explicitly enabled by {\tt glutPostOverlayRedisplay}
   or implicitly in response to overlay redisplay events from the window system.
\item[\takeNote]  Set when the window system reports a region of the window's normal plane
   is undefined (for example, damaged by another window moving or being initially shown).
   The specifics of when damage occurs are left to the window system to determine.
   The window's {\em redisplay} state is always set true when damage occurs.  {\em normalDamaged}
   is cleared whenever the window's display callback returns.
\item[\takeNote]  Set when the window system reports a region of the window's overlay plane
   is undefined (for example, damaged by another window moving or being initially shown).
   The specifics of when damage occurs are left to the window system to determine.  The
   damage may occur independent from damage to the window's normal plane.
   The window's {\em redisplay} state is always set true when damage occurs.  {\em normalDamaged}
   is cleared whenever the window's display callback returns.
\end{description}
}

When an overlay is established, {\em overlay} frame buffer capability state is maintained
as described in Section \ref{frame-buffer-state}.  The {\em layerInUse} determines
whether {\tt glutGet} returns normal plane or overlay state when an overlay is
established.

\subsection{Menu State}

\resetNote

{\footnotesize
\begin{tabular}{|l|l|l|l|l|} \hline
Name & Type & Set/Change & Get & Initial \\ \hline
number & Integer & - & glutSetMenu & {\em top-level:} glutCreateMenu \takeNote \\
select & Callback & - & - & glutCreateMenu \\
items & list of MenuItem & - & - & - \\
numItems & Integer & - & GLUT\_MENU\_NUM\_ITEMS & 0 \\
\hline
\end{tabular}
}
\resetNote

{\footnotesize
\begin{description}
\itemsep 0in
\item[\takeNote]  Assigned dynamically from unassigned window numbers greater than zero.
\end{description}
}

